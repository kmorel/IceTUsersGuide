% -*- latex -*-


\chapter{Communicators}
\label{chap:Communicators}

\IceT implements an abstract communication layer.  As we will see later in
this chapter, this communication layer is a message passing interface based
heavily on \MPI.\footnote{In fact, the original implementation of \IceT
  used \MPI directly.  The abstract layer was inserted later as a
  more-or-less cut-and-paste operation.}  As an end user to \IceT, you need
to know almost nothing about this communication layer.  You need only to
get a reference to an \CType{IceTCommunicator} object.  This object is
opaque.  You only need to get one, pass it to
\index{context!\IceT}\CFunc{icetCreateContext}, and then delete it.
\CFunc{icetCreateContext} will duplicate the communicator, so you need not
worry about when you delete the context you created.

Most of the time you will use the built-in \MPI implementation of the
communicator, which is discussed in the first section.  If necessary, you
can write your own communicator, which is discussed in the following section.

\section{MPI Communicators}
\label{sec:Communicators:MPI_Communicators}

Using the \MPI implementation of a communicator, you simply include
\index{IceTMPI.h}\textC{IceTMPI.h} in your source and link
\index{IceTMPI (library)}IceTMPI into your own library or executable.  The
only function you need to use is \CFunc{icetCreateMPICommunicator}.

\begin{Table}{3}
  \multicolumn{3}{l}{\CType{IceTCommunicator}\textC{ }\CFunc{icetCreateMPICommunicator}\textC{(}}\\
  \makebox[3.5in]{}&\CType{MPI\_Comm}&\CArg{mpi\_comm}\quad\textC{);}
\end{Table}

Quite simply, \CFunc{icetCreateMPICommunicator} converts an
\CType{MPI\_Comm}, an \MPI communicator, into an \CType{IceTCommunicator},
an \IceT communicator.  \CFunc{icetCreateMPICommunicator} duplicates the
\MPI communicator.  Thus, you can delete the \CArg{mpi\_comm} communicator
as soon as \CFunc{icetCreateMPICommunicator} exits.  Furthermore, the
returned \CType{IceTCommunicator} will internally manage the \MPI
communicator it created.

Once created, the \CType{IceTCommunicator} may be deleted with
\CFunc{icetDestroyMPICommunicator}.

\begin{Table}{3}
  \textC{void }\CFunc{icetDestroyMPICommunicator}\textC{(}&\CType{IceTCommunicator}&\CArg{comm}\quad\textC{);}
\end{Table}

\CFunc{icetDestroyMPICommunicator} will release all the resources used by
\CArg{comm}.  This includes the internal \MPI communicator, which you do
not have direct access to.  \CArg{comm} will be invalid once you call
\CFunc{icetDestroyMPICommunicator}.  However, you do not have to worry
about any \IceT context you have passed it to since they will have
duplicated the communicator.

Using the \MPI communicator is easy.  First, you include the
\index{IceTMPI.h}\textC{IceTMPI.h} header.

\index{IceT.h}
\begin{code}
#include <IceT.h>
#include <IceTMPI.h>
\end{code}

When you are ready to create an \IceT context (usually during the
initialization of your program), create the \MPI-based communicator, use it
to initialize the context, and then destroy the communicator.

\index{context!\IceT}
\index{icetCreateMPICommunicator}
\index{icetCreateContext}
\index{icetDestroyMPICommunicator}
\begin{code}
  icetComm = icetCreateMPICommunicator(MPI_COMM_WORLD);
  icetContext = icetCreateContext(icetComm);
  icetDestroyMPICommunicator(icetComm);
\end{code}

Once you have a context, you can use \IceT as explained throughout this
document.  When you are ready, destroy the context as you normally would.

\index{icetDestroyContext}
\begin{code}
  icetDestroyContext(icetContext);
\end{code}

Finally, do not forget to use the \index{IceTMPI (library)}IceTMPI
library when linking your executable or library.

A more detailed example of using the \MPI communicator is in the
Chapter~\ref{chap:Tutorial} tutorial.

\section{User Defined Communicators}
\label{sec:Communicators:User_Defined_Communicators}

Occasionally, it may be necessary to provide your own version of a parallel
communicator.  This may be because you are using a communication library
other than \MPI.  It may also be because you wish to augment the behavior
of \MPI when it is used by \IceT.  To provide your own communicator, you
need only to create an \CType{IceTCommunicator} object.  In previous
sections we have discussed \CType{IceTCommunicator} as an opaque type, and
unless you are implementing your own you should treat it as such.  If you
are implementing an \CType{IceTCommunicator}, you will see that it is simply
a pointer to a structure containing references to several communication
functions.

\begin{code}
typedef struct IceTCommRequestStruct {
    IceTEnum magic_number;
    IceTVoid *internals;
} *IceTCommRequest;
#define ICET_COMM_REQUEST_NULL ((IceTCommRequest)NULL)

struct IceTCommunicatorStruct {
    struct IceTCommunicatorStruct *
         (*Duplicate)(struct IceTCommunicatorStruct *self);
    void (*Destroy)(struct IceTCommunicatorStruct *self);
    struct IceTCommunicatorStruct *
         (*Subset)(struct IceTCommunicatorStruct *self,
                   int count,
                   const IceTInt32 *ranks);
    void (*Barrier)(struct IceTCommunicatorStruct *self);
    void (*Send)(struct IceTCommunicatorStruct *self,
                 const void *buf,
                 int count,
                 IceTEnum datatype,
                 int dest,
                 int tag);
    void (*Recv)(struct IceTCommunicatorStruct *self,
                 void *buf,
                 int count,
                 IceTEnum datatype,
                 int src,
                 int tag);

    void (*Sendrecv)(struct IceTCommunicatorStruct *self,
                     const void *sendbuf,
                     int sendcount,
                     IceTEnum sendtype,
                     int dest,
                     int sendtag,
                     void *recvbuf,
                     int recvcount,
                     IceTEnum recvtype,
                     int src,
                     int recvtag);
    void (*Gather)(struct IceTCommunicatorStruct *self,
                   const void *sendbuf,
                   int sendcount,
                   IceTEnum datatype,
                   void *recvbuf,
                   int root);
    void (*Gatherv)(struct IceTCommunicatorStruct *self,
                    const void *sendbuf,
                    int sendcount,
                    IceTEnum datatype,
                    void *recvbuf, 
                    const int *recvcounts,
                    const int *recvoffsets,
                    int root);
    void (*Allgather)(struct IceTCommunicatorStruct *self,
                      const void *sendbuf,
                      int sendcount,
                      IceTEnum datatype,
                      void *recvbuf);
    void (*Alltoall)(struct IceTCommunicatorStruct *self,
                     const void *sendbuf,
                     int sendcount,
                     IceTEnum datatype,
                     void *recvbuf);

    IceTCommRequest (*Isend)(struct IceTCommunicatorStruct *self,
                             const void *buf,
                             int count,
                             IceTEnum datatype,
                             int dest,
                             int tag);
    IceTCommRequest (*Irecv)(struct IceTCommunicatorStruct *self,
                             void *buf,
                             int count,
                             IceTEnum datatype,
                             int src,
                             int tag);

    void (*Wait)(struct IceTCommunicatorStruct *self, IceTCommRequest *request);
    int  (*Waitany)(struct IceTCommunicatorStruct *self,
                    int count, IceTCommRequest *array_of_requests);

    int  (*Comm_size)(struct IceTCommunicatorStruct *self);
    int  (*Comm_rank)(struct IceTCommunicatorStruct *self);
    void *data;
};

typedef struct IceTCommunicatorStruct *IceTCommunicator;
\end{code}

To create a custom \CType{IceTCommunicator} simply allocate the structure
and fill in the function pointers.  An implementation for a function that
creates an \IceT communicator might look like the following.  In this
example, the \textC{my*} functions are implementations of the communication
functions. 

\begin{code}
IceTCommunicator myCreateCommunicator(myCommType myComm)
{
  IceTCommunicator comm = malloc(sizeof(struct IceTCommunicatorStruct));

  comm->Duplicate = myDuplicate;
  comm->Destroy = myDestroy;
  comm->Send = mySend;
  /* And so on... */

  comm->data = malloc(sizeof(myComm))
  /* Making a duplicate here would be better. */
  memcpy(comm->data, myComm, sizeof(myComm));

  return comm;
}
\end{code}

The paired destruction function should probably just call the Destroy
function of the communicator (or vice versa) to ensure that destroy works
either way.

\begin{code}
void myDestroyCommunicator(IceTCommunicator comm)
{
  comm->Destroy(comm);
}
\end{code}

\begin{code}
static void myDestroy(IceTCommunicator self)
{
  myCommType *myComm = (myCommType *)self->data;
  /* Release resources of myComm. */
  free(myComm);
  free(self);
}
\end{code}

For a more concrete example of implementing an \IceT communicator, see the
\IceT code for the \MPI communicator.
