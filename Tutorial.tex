% -*- latex -*-

\chapter{Tutorial}
\label{chap:Tutorial}

In this chapter we outline the steps required to create a simple \IceT
application from building the \IceT source, using the created libraries,
and writing your own applications.  \IceT is solely responsible for the
image composition part of parallel rendering.  Thus, it relies on two other
APIs: \index{OpenGL}\keyterm{OpenGL} for rendering and a communication
layer for passing messages such as \index{MPI}\keyterm{MPI}, the Message
Passing Interface.  Both have implementations in nearly every computer
architecture.

This tutorial assumes the reader is familiar with OpenGL.  If this is your
first experience with OpenGL programing, consider trying some typical
serial rendering before jumping into the parallel rendering domain.  A
familiarity with MPI is also helpful.

\section{Building \IceT}
\label{sec:Tutorial:Building_IceT}

The \IceT build process is very portable.  It can be compiled on Microsoft
Windows, Macintosh OS X, and a wide variety of Unix implementations.  \IceT
can be built on just about any platform that has an
\index{OpenGL}OpenGL~1.1 compliant installation.  Most modern operating
systems come distributed with OpenGL.  For those that are not, you can
usually use the \index{Mesa~3D}\keyterm{Mesa 3D} library
(\href{www.mesa3d.org}{www.mesa3d.org}), a software implementation of
OpenGL.  An installation of MPI is also almost always needed, although not
strictly required.  \index{MPICH}\keyterm{MPICH}
(\href{http://www-unix.mcs.anl.gov/mpi/mpich2/}{http://www-unix.mcs.anl.gov/mpi/mpich2/})
is a free and widely portable implementation of MPI.

\IceT uses \index{CMake}\keyterm{CMake} to build across so many different
platforms.  As such, you will have to download the CMake build tools from
\href{www.cmake.org}{www.cmake.org} and install.  Then, create a build
directory and run the CMake program (from the ``Start'' menu on Windows or
ccmake on Unix and Mac OS X).  CMake will determine the parameters of your
system and do its best to find libraries on which \IceT depends.  The CMake
program, shown in Figure~\ref{fig:Tutorial:ccmake} will also provide a GUI
to allow you to easily change build parameters and external libraries.
\begin{figure}
  \centering
  \includegraphics[height=1.5in]{images/ccmakeUnix}
  \qquad
  \includegraphics[height=1.5in]{images/ccmakeWindows}
  \caption[CMake user interface.]{The CMake user interface.  The Unix
    version is on the left whereas the Microsoft Windows version is on the
    right.}
  \label{fig:Tutorial:ccmake}
\end{figure}

CMake will generate a set of build files for the local system.  The type of
files depends on the type of machine you are using and the compile system
you have chosen to use.  On Unix machines, make files are the most common.
On Windows, you usually generate MSVC project files or nmake files.  On Mac
OS X, either make files or Xcode project files are commonly generated based
on user selection.  You then use the native build system to build and,
optionally, install \IceT.

\section{Linking to \IceT Libraries}
\label{sec:Tutorial:Linking_to_IceT_Libraries}

\IceT comes with three libraries: \index{icet (library)}\keyterm{icet},
\index{icet\_strategies~(library)}\keyterm{icet\_strategies}, and
\index{icet\_mpi~(library)}\keyterm{icet\_mpi}.  The actual filenames of
these libraries varies depending on the filesystem and build type.  For
example, on most Unix systems, a static build results in filenames of
\index{libicet.a}libicet.a and the like whereas shared libraries are
libicet.so.  Windows has libraries with names like icet.lib as well as
icet.dll if building shared libraries.  However, the difference in these
filenames usually hidden by the build system, especially if you use a
portable build system like \index{CMake}CMake.

You are, of course, free to use whatever build system you like, whether it
be system specific or cross platform.  Using \IceT is simply a matter of
finding the header and library files.  However, because \IceT is built with
CMake, it comes with some extra facilities for helping other CMake builds
find it.  This section will give you the bare minimum you need to set up
CMake to build an application using \IceT.  Readers interested more about
CMake should pick up a copy of \emph{Mastering CMake} by Ken Martin and
Bill Hoffman.

You define a build system with CMake by creating a
\index{CMakeLists.txt}\keyterm{CMakeLists.txt} file.  The CMakeLists.txt
file is basically a simple script that gives commands the CMake to tell it
how to build your project.  Most CMakeLists.txt files start with the
\textC{PROJECT} command, which associates a name with your project and
optionally specifies a language.
\begin{code}
PROJECT(IceT_Tutorial)
\end{code}

Distributed with the \IceT source code is a file called
\index{FindIceT.cmake}\keyterm{FindIceT.cmake} that provides all the CMake
facilities needed to find and use an \IceT build.  It works by finding a
file called \index{ICETConfig.cmake}\keyterm{ICETConfig.cmake}, which is
written when \IceT is built and contains all the necessary build settings.
The FindIceT.cmake script can be invoked with the \textC{FIND\_PACKAGE}
command.  After the \IceT package is found, a variable named
\textC{ICET\_USE\_FILE} is set to a file that may be \textC{INCLUDE}d in
your project to point it to the directories containing the header and
library files.
\begin{code}
FIND_PACKAGE(IceT REQUIRED)
INCLUDE(${ICET_USE_FILE})
\end{code}
%$

Any application using \IceT will also be using \index{OpenGL}OpenGL and
almost all will be using \index{MPI}MPI.  In addition, the example in the
following section also uses \index{GLUT}GLUT for window management.  CMake
comes with modules to find all three of these libraries, which makes it
easy to include in our project.
\begin{code}
FIND_PACKAGE(OpenGL REQUIRED)
FIND_PACKAGE(GLUT REQUIRED)
FIND_PACKAGE(MPI REQUIRED)

MARK_AS_ADVANCED(CLEAR
  MPI_INCLUDE_PATH
  MPI_LIBRARY
  MPI_EXTRA_LIBRARY
  )

INCLUDE_DIRECTORIES(
  ${OPENGL_INCLUDE_DIR}
  ${MPI_INCLUDE_PATH}
  ${GLUT_INCLUDE_DIR}
  )
\end{code}
%$

The only think left to do is to tell CMake to build a program from a set of
sources and libraries specified with the \textC{ADD\_EXECUTABLE} and
\textC{TARGET\_LINK\_LIBRARIES} commands, respectively.
\begin{code}
ADD_EXECUTABLE(Tutorial Tutorial.c)
TARGET_LINK_LIBRARIES(Tutorial
  ${OPENGL_LIBRARIES}
  ${GLUT_LIBRARIES}
  ${MPI_LIBRARY}
  ${MPI_EXTRA_LIBRARY}
  ${ICET_CORE_LIBS}
  ${ICET_MPI_LIBS}
  )
\end{code}

\section{Creating \IceT Enabled Applications}
\label{sec:Tutorial:Creating_IceT_Enabled_Applications}

To use \IceT, include it's header:
\index{ice-t.h}\index{GL/ice-t.h}\textC{GL/ice-t.h}.  You will almost
always need to also include the header containing an MPI version of an
\IceT communicator:
\index{ice-t\_mpi.h}\index{GL/ice-t\_mpi.h}\textC{GL/ice-t\_mpi.h}.  On the
rare occasion that you need to use \IceT with a communication layer other
than MPI, you can define a custom communicator as described in
Chapter~\ref{chap:Communicators}.
\begin{code}
#include <GL/ice-t.h>
#include <GL/ice-t_mpi.h>
\end{code}

Before you call any \IceT functions, you need to initialize MPI by calling
\textC{MPI\_Init}\index{MPI\_Init}.  You will also need to create an OpenGL
context (that is, open an OpenGL window).  Do this by first creating an
\IceT \index{communicator}\keyterm{communicator} from an MPI communicator
and then using that to create an
\index{context!\IceT}{\keyterm{\IceT context}.
\index{icetCreateMPICommunicator}
\index{icetCreateContext}
\begin{code}
    comm = icetCreateMPICommunicator(MPI_COMM_WORLD);
    context = icetCreateContext(comm);
\end{code}
In the proceeding code, \textC{comm} is of type
\index{IceTCommunicator}\textC{IceTCommunicator} and \textC{context} is of
type \index{IceTContext}\textC{IceTContext}.

Now that we have created and activated an \IceT communicator, as well as
initialized the \IceT \index{state}\keyterm{state}, we can start using
\IceT.  It is often useful to first query \IceT on the size of the parallel
job it is running in and what is the local process id, or
\index{rank}\keyterm{rank}.  The values are stored in variables of type
\textC{GLint}.
\begin{code}
    icetGetIntegerv(ICET_RANK, &rank);
    icetGetIntegerv(ICET_NUM_PROCESSES, &num_proc);
\end{code}

In addition to an \IceT context, you will also need an
\index{context!\OpenGL}\keyterm{\OpenGL context}.  In other words, you need
to make the rendering window in which the \OpenGL rendering commands will
go.  The process for doing this is greatly dependent on the windowing
system and beyond the scope of this document.  It is usually easiest to use
a third party API to do this.  If you are not already using a GUI tool that
generates \OpenGL windows for you, then the \index{GLUT}\keyterm{GLUT} API
is a popular choice for simple applications.

Before rendering, we need to tell \IceT the layout of the tile display
using the \CFunc{icetResetTiles} and \CFunc{icetAddTile} functions.  These
commands must be executed with the same arguments on all processes of the
parallel job.  \IceT will assume that you setup the same display layout
everywhere.

If you are not actually driving a tile display and instead just generating
a desktop-sized image, the following commands will correctly establish the
\IceT state.
\begin{code}
    icetResetTiles();
    icetAddTile(0, 0, WINDOW_WIDTH, WINDOW_HEIGHT, 0);
\end{code}

The \CFunc{icetResetTiles} function simply tells \IceT that you are about
to define a display layout.  Each call to \CFunc{icetAddTile} defines a
tile in the display.  In the case of a single image, the
\index{single-tile~rendering}\keyterm{single-tile rendering} mode,
\CFunc{icetAddTile} is called only once.  The first two arguments to
\CFunc{icetAddTile} have no effect in this mode.  The third and fourth
arguments are the width and height of the image to create.  Usually you set
this to the width and the height of the display window, but the
\hyperref[sec:Customizing_Compositing:Image_Inflation]{Image Inflation}
section in Chapter~\ref{chap:Customizing_Compositing} describes other usage
for these parameters.  The final argument is the rank of the
\index{display~process}\keyterm{display process}.  After a rendering the
final complete image will available only on this process.  In the example
above, we have direct the image to go to process zero, often referred to as
the \index{root~process}\keyterm{root process}.

To define an actual tile display, simply call the \CFunc{icetAddTile}
function multiple times.  When describing tiles in a display, the first two
arguments of \CFunc{icetAddTile} describe where the lower left corner of
the tile is located in respect to the overall display.  All together, the
first four arguments specify a viewport for the tile in an a single,
cohesive high resolution display (which is what we are trying to achieve
with our tile display).  The code below defines a $2 \times 2$ tile display
with the top two tiles displayed by processes 0 and 1 and the bottom two
tiles displayed by processes 2 and 3.
\begin{code}
    icetResetTiles();
    icetAddTile(0,           WINDOW_HEIGHT, WINDOW_WIDTH, WINDOW_HEIGHT, 0);
    icetAddTile(WINDOW_WIDTH,WINDOW_HEIGHT, WINDOW_WIDTH, WINDOW_HEIGHT, 1);
    icetAddTile(0,           0,             WINDOW_WIDTH, WINDOW_HEIGHT, 2);
    icetAddTile(WINDOW_WIDTH,0,             WINDOW_WIDTH, WINDOW_HEIGHT, 3);
\end{code}

\IceT contains several \index{strategy}\keyterm{strategies} for image
composition.  Changing the strategy modifies the algorithm \IceT uses for
parallel image compositing.  You need to tell \IceT which strategy to use
with the \CFunc{icetStrategy} function.  The code below sets \IceT to use
the \index{strategy!reduce}\keyterm{reduce strategy}, which has proven to
be an all-around good performer.
\begin{code}
    icetStrategy(ICET_STRATEGY_REDUCE);
\end{code}
Like with the display set up, all processes must set the same strategy.

\IceT is almost ready to go.  We just need to tell it some minimal
information about how to render your geometry.  First, \IceT needs to know
the spatial extent of the geometry to be drawn (in object space).  The most
natural way to do this is to use the \CFunc{icetBoundingBox} function,
which defines an axis-aligned box defined by the minimum and maximum
coordinates in each dimension.
\begin{code}
    icetBoundingBoxf(x_min, x_max, y_min, y_max, z_min, z_max);
\end{code}
The parameters can, and should be, different on each process, since each
process will have a different partition of data.  Strictly speaking,
identifying the geometry bounds is not necessary.  If they are not defined,
\IceT will assume the geometry covers the entire screen.  When rendering a
single small image, the information is of little consequence.  However,
when rendering larger images this information can dramatically improve the
performance of image composting.  Specifying the bounds can be critical on
large tile displays.

The second and final piece of information \IceT needs is a way to draw your
geometry.  \IceT achieves this through a
\index{drawing~callback}\index{callback|see{drawing~callback}}\index{rendering~callback|see{drawing~callback}}\keyterm{drawing
  callback}.
\begin{code}
    icetDrawFunc(drawScene);
\end{code}
The drawing callback is a pointer to any function that issues
\OpenGL commands that render geometry to the active frame buffer.  The
callback is free to issue most \OpenGL so long as it restores all the
\OpenGL state (except, of course, frame buffer contents).  Also, the
callback function should modify neither the projection matrix nor the
clear color.  Care needs to be taken if the callback modifies the model view
matrix.  More details are given in the
\hyperref[sec:Basic_Usage::Drawing_Callback]{Drawing Callback} section of
Chapter~\ref{chap:Basic_Usage}.

\IceT is now ready to render.  Rendering is initiated with a call to
\CFunc{icetDrawFrame}.  The \CFunc{icetDrawFrame} must be called on all
processes.  The function will render the scene using the provided
\index{drawing~callback}drawing callback, composite the image, and place
the appropriate images in the back \OpenGL buffers of the appropriate
\index{display~process}display processes.
\begin{code}
    icetDrawFrame();
\end{code}

Parallel rendering is now enabled in your application.  Simply call
\CFunc{icetDrawFrame} every time you wish to draw a new image.  The
geometry rendered by your \index{drawing~callback} may change from frame to
frame so long as you ensure that you also update \IceT with the bounds of
your geometry if it changes.

The following code is a full example of a simple \IceT application.  Do not
be alarmed by the length.  The majority of the code is spent in setting up
the supporting libraries (OpenGL, GLUT, and MPI) and in comments.

\codeinput{examples/Tutorial.c}
