% -*- latex -*-


\setDate{August  9, 2010}
% The following must all be on one line for latex2man to work.
\begin{Name}{3}{icetImageCopyColor}{Kenneth Moreland}{\IceT Reference}{\icetImageCopyColor, \icetImageCopyDepth -- retrieve pixel data from image}

  \mansection{Synopsis}

  \textC{\#include <IceT.h>}

  \begin{Table}{4}
    \textC{void}&\icetImageCopyColorub\textC{(}&\textC{const }\CType{IceTImage}&\CArg{image}\textC{,} \\
    &&\textC{IceTUByte *}&\CArg{color\_buffer}, \\
    &&\textC{IceTEnum}&\CArg{color\_format}\quad\textC{);} 
  \end{Table}

  \begin{Table}{4}
    \textC{void}&\icetImageCopyColorf\textC{(}&\textC{const }\CType{IceTImage}&\CArg{image}\textC{,} \\
    &&\textC{IceTFloat *}&\CArg{color\_buffer}, \\
    &&\textC{IceTEnum}&\CArg{color\_format}\quad\textC{);}
  \end{Table}

  \begin{Table}{4}
    \textC{void}&\icetImageCopyDepthf\textC{(}&\textC{const }\CType{IceTImage}&\CArg{image}\textC{,} \\
    &&\textC{IceTFloat *}&\CArg{depth\_buffer}, \\
    &&\textC{IceTEnum}&\CArg{depth\_format}\quad\textC{);}
  \end{Table}

  
  \mansection{Description}

  The \icetImageCopyColor suite of functions retrieve color data from images
  and the \icetImageCopyDepth functions retrieve depth data from images.
  Each function takes a pointer to an existing buffer that must be large
  enough to hold all pixels in the image.  The data from the images is
  copied into these buffers, performing format conversions as necessary.
  Because data is copied into the provided buffer, subsequently changing
  values in the buffer has no effect on the image object (as opposed to the
  behavior of \icetImageGetColor and \icetImageGetDepth).

  The pixel data is always tightly packed in horizontal major order.  Color
  data that comprises tuples such as RGBA have the components for each
  pixel packed together in that order.  The first entry in the array
  corresponds to the pixel in the lower left corner of the image.  The next
  entry is immediately to the right of the first pixel, and so on.  The
  dimensions of the array can be retrieved with the \icetImageGetWidth and
  \icetImageGetHeight functions.

  Each of these functions provides a typed version of the image data array.
  They can only succeed if the type the request matches the type specified
  by the \CArg{color\_format} or \CArg{depth\_format} argument.  It is an
  error, for example, to request unsigned byte color data for a floating
  point color format.  Although specifying the format may be redundant (it
  could be implied by the type being retrieved), \IceT requires it for
  completeness and to support possible future data formats.

  Use \icetImageCopyColorub to retrieve an array of 8-bit unsigned bytes.
  Using this function is only valid if \CArg{color\_format} is
  \CEnum{ICET\_IMAGE\_COLOR\_RGBA\_UBYTE}.

  Use \icetImageCopyColorf to retrieve an array of floating point color
  values.  Using this function is only valid if \CArg{color\_format} is
  \CEnum{ICET\_IMAGE\_COLOR\_RGBA\_FLOAT}.

  Use \icetImageGetDepthf to retrieve an array of floating point depth
  values.  Using this function is only valid if \CArg{depth\_format} is
  \CEnum{ICET\_IMAGE\_DEPTH\_FLOAT}.


  \mansection{Errors}

  \begin{Description}[ICET\_INVALID\_OPERATION]
  \item[\CErrorEnum{ICET\_INVALID\_OPERATION}]
    The \CArg{image} object does not have a color or depth buffer from
    which to copy data.
  \item[\CErrorEnum{ICET\_INVALID\_ENUM}]
    The requested \CArg{color\_format} or \CArg{depth\_format} is
    incompatible with the type of the buffer.
  \end{Description}


  \mansection{Warnings}

  None.


  \mansection{Bugs}

  None known.


  \mansection{Copyright}
  Copyright \copyright 2010 Sandia Corporation

  @copyright@

  \mansection{See Also}

  \CFuncSeeAlso{icetImageGetColor},
  \CFuncSeeAlso{icetImageGetDepth}

\end{Name}


% These are emacs settings that go at the end of the file.
% Local Variables:
% writestamp-format: "%B %e, %Y"
% writestamp-prefix: "^\\\\setDate{"
% writestamp-suffix: "}$"
% End:
