% -*- latex -*-


\setDate{August  9, 2010}
% The following must all be on one line for latex2man to work.
\begin{Name}{3}{icetGLInitialize}{Kenneth Moreland}{\IceT Reference}{\CFunc{icetGLInitialize} -- initialize the \IceT \OpenGL layer}

  \mansection{Synopsis}

  \textC{\#include <IceTGL.h>}

  \begin{Table}{3}
    \textC{void }\CFunc{icetGLInitialize}\textC{(}&\textC{void}&\textC{);}
  \end{Table}

  
  \mansection{Description}

  Initializes the \OpenGL layer of \IceT.  \CFunc{icetGLInitialize} must be
  called before any other function starting with \textC{icetGL} (except
  \CFunc{icetGLIsInitialized}).

  Management for the \OpenGL layer is held in the state of the current
  \IceT context.  Thus, \CFunc{icetGLInitialize} must be called once per
  \IceT context.  If you are using a context for rendering with \OpenGL, it
  is recommended that you call \CFunc{icetGLInitialize} immediately after
  calling \CFunc{icetCreateContext}.


  \mansection{Errors}

  None.


  \mansection{Warnings}

  \begin{Description}[ICET\_INVALID\_OPERATION]
  \item[\CErrorEnum{ICET\_INVALID\_OPERATION}]
    \CFunc{icetGLInitialize} is called twice for the same context.
  \end{Description}


  \mansection{Bugs}

  None known.


  \mansection{Copyright}
  Copyright \copyright 2010 Sandia Corporation

  @copyright@

  \mansection{See Also}

  \CFuncSeeAlso{icetCreateContext},
  \CFuncSeeAlso{icetGLIsInitialized}

\end{Name}


% These are emacs settings that go at the end of the file.
% Local Variables:
% writestamp-format: "%B %e, %Y"
% writestamp-prefix: "^\\\\setDate{"
% writestamp-suffix: "}$"
% End:
