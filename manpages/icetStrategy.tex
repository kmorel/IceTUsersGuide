% -*- latex -*-


\setDate{August  9, 2010}
% The following must all be on one line for latex2man to work.
\begin{Name}{3}{icetStrategy}{Kenneth Moreland}{\IceT Reference}{\CFunc{icetStrategy} -- set the strategy used to composite images.}

  \mansection{Synopsis}

  \textC{\#include <IceT.h>}

  \begin{Table}{3}
    \textC{void }\CFunc{icetStrategy}\textC{(}&\textC{IceTEnum}&\CArg{strategy}\quad\textC{);}
  \end{Table}

  
  \mansection{Description}

  The \IceT API comes packaged with several algorithms for compositing
  images.  The algorithm to use is determined by selecting a
  \CArg{strategy}.  The strategy is selected with \CFunc{icetStrategy}.  A
  strategy must be selected before \CFunc{icetDrawFrame} is called.

  A strategy is chosen from one of the following provided enumerated
  values:

  @enum_strategy@

  Not all of the strategies support ordered image composition.
  \CEnum{ICET\_STRATEGY\_SEQUENTIAL}, \CEnum{ICET\_STRATEGY\_DIRECT}, and
  \CEnum{ICET\_STRATEGY\_REDUCE} do support ordered image composition.
  \CEnum{ICET\_STRATEGY\_SPLIT} and \CEnum{ICET\_STRATEGY\_VTREE} do not
  support ordered image composition and will ignore
  \CEnum{ICET\_ORDERED\_COMPOSITE} if it is enabled.

  Some of the strategies, namely \CEnum{ICET\_STRATEGY\_SEQUENTIAL} and
  \CEnum{ICET\_STRATEGY\_REDUCE}, use a sub-strategy that composites the
  image for a single tile.  This single image strategy can also be
  specified with \CFunc{icetSingleImageStrategy}.


  \mansection{Errors}

  \begin{Description}[ICET\_INVALID\_OPERATION]
  \item[\CErrorEnum{ICET\_INVALID\_ENUM}]
    The \CArg{strategy} argument does not represent a valid strategy.
  \end{Description}


  \mansection{Warnings}

  None.


  \mansection{Bugs}

  None known.


  \mansection{Copyright}
  Copyright \copyright 2003 Sandia Corporation

  @copyright@

  \mansection{See Also}

  \CFuncSeeAlso{icetDrawFrame},
  \CFuncSeeAlso{icetGetStrategyName}
  \CFuncSeeAlso{icetSingleImageStrategy}

\end{Name}


% These are emacs settings that go at the end of the file.
% Local Variables:
% writestamp-format: "%B %e, %Y"
% writestamp-prefix: "^\\\\setDate{"
% writestamp-suffix: "}$"
% End:
