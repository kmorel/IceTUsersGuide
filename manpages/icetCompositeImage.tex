% -*- latex -*-


\setDate{October  9, 2014}
% The following must all be on one line for latex2man to work.
\begin{Name}{3}{icetCompositeImage}{Kenneth Moreland}{\IceT Reference}{\CFunc{icetCompositeImage} -- composites a pre-rendered image}

  \mansection{Synopsis}

  \textC{\#include <IceT.h>}

  \begin{Table}{3}
    %@% IF notlatex2man %@%
    \multicolumn{3}{l}{
    %@% END-IF %@%
      \CType{IceTImage}\textC{ }\CFunc{icetCompositeImage}\textC{(}
    %@% IF notlatex2man %@%
    } \\
    \qquad\qquad\qquad\qquad\qquad\qquad\qquad\qquad
    %@% END-IF %@%
    &\textC{const IceTVoid *}&\CArg{color\_buffer}\textC{,} \\
    &\textC{const IceTVoid *}&\CArg{depth\_buffer}\textC{,} \\
    &\textC{const IceTInt *}&\CArg{valid\_pixels\_viewport}\textC{,} \\
    &\textC{const IceTDouble *}&\CArg{projection\_matrix}\textC{,} \\
    &\textC{const IceTDouble *}&\CArg{modelview\_matrix}\textC{,} \\
    &\textC{const IceTFloat *}&\CArg{background\_color}\quad\textC{);}
  \end{Table}

  
  \mansection{Description}

  The \CFunc{icetCompositeImage} function takes image buffer data and
  composites it to a single image. This function behaves similarly to
  \CFunc{icetDrawFrame} except that instead of using callback functions to
  render and retrieve image data, the images are pre-rendered and passed
  directly to \CFunc{icetCompositeImage}. Although it is more efficient to
  allow \IceT to determine rendering projections and use callbacks, it is
  often more convenient for applications to integrate \IceT as a separate
  compositing step after the rendering.

  Before \IceT may composite an image, the display needs to be defined
  (using \CFunc{icetAddTile}), the buffer formats need to be specified
  (using \icetSetColorFormat and \icetSetDepthFormat), and the composite
  strategy must be set (using \CFunc{icetStrategy}). The single image
  sub-strategy may also optionally be set (using
  \CFunc{icetSingleImageStrategy}).

  All process must call \CFunc{icetCompositeImage} for the operation to
  complete on any process in a parallel job.

  The \CArg{color\_buffer} and \CArg{depth\_buffer} arguments point to
  memory buffers that contain the image data. The image data is always
  stored in densely packed arrays in row-major order (a.k.a. x-major order
  or by scan-lines). The first horizontal scan-line is at the bottom of the
  image with subsequent scan-lines moving up. The size of each image buffer
  is expected to be the width and the height of the global viewport (which
  is set indirectly with \CFunc{icetAddTile}). The global viewport is
  stored in the \CEnum{ICET\_GLOBAL\_VIEWPORT} state variable. If only one
  tile is specified, then the width and height of the global viewport will
  be the same as this one tile.

  The format for \CArg{color\_buffer} is expected to be the same as what is
  set with \icetSetColorFormat. The following formats and their
  interpretations with respect to \CArg{color\_buffer} are as follows.

  @enum_image_color@

  Likewise, the format for \CArg{depth\_buffer} is expected to be the same
  as what is set with \icetSetDepthFormat. The following formats and
  their interpretations with respect to \CArg{depth\_buffer} are as
  follows.

  @enum_image_depth@

  If the current format does not have a color or depth, then the respective
  buffer argument should be set to \textC{NULL}.

  Care should be taken to make sure that the color and depth buffer formats
  are consistent to the formats expected by \IceT. Mismatched formats will
  result in garbage images and possible memory faults.

  Also note that when compositing with color blending
  (\CFunc{icetCompositeMode} is set to
  \CEnum{ICET\_COMPOSITE\_MODE\_BLEND}), the color buffer must be rendered
  with a black background in order for the composite to complete
  correctly. A colored background can later be added using the
  \CArg{background\_color} as described below.

  \CArg{valid\_pixels\_viewport} is an optional argument that makes it
  possible to specify a subset of pixels that are valid. In parallel
  rendering it is common for a single process to render geometry in only a
  small portion of the image, and \IceT can take advantage of this
  information. If the rendering system identifies such a region, it can be
  specified with \CArg{valid\_pixels\_viewport}.

  Like all viewports in \IceT, \CArg{valid\_pixels\_viewport} is an array
  of 4 integers in the form $\langle x, y, width, height \rangle$. This
  viewport is given in relation to the image passed in the
  \CArg{color\_buffer} and \CArg{depth\_buffer} arguments. Everything
  outside of this rectangular region will be ignored. For example, if the
  \CArg{valid\_pixels\_viewport} $\langle 10, 20, 150, 100 \rangle$ is
  given, then \CFunc{icetCompositeImage} will ignore all pixels in the
  bottom $10$ rows, the left $20$ columns, anything above the
  $160^\mathrm{th}$ ($10+150$) row, and anything to the right of the
  $120^\mathrm{th}$ ($20+100$) column.

  If \CArg{valid\_pixels\_viewport} is \textC{NULL}, then all pixels in the
  input image are assumed to be valid.

  \CArg{projection\_matrix} and \CArg{modelview\_matrix} are optional
  arguments that specify the projection that was used during rendering.
  When applied to the geometry bounds information given with
  \CFunc{icetBoundingBox} or \CFunc{icetBoundingVertices}, this provides
  \IceT with further information on local image projections. If the given
  matrices are not the same used in the rendering or the given bounds do
  not contain the geometry, \IceT may clip the geometry in surprising ways.
  If these arguments are set to \textC{NULL}, then geometry projection will
  not be considered when determining what parts of images are valid.

  The \CArg{background\_color} argument specifies the desired background
  color for the image. It is given as an array of 4 floating point values
  specifying, in order, the red, green, blue, and alpha channels of the
  color in the range from $0.0$ to $1.0$.

  When rendering using a depth buffer, the background color is used to fill
  in empty regions of images. When rendering using color blending, the
  background color is used to correct colored backgrounds.

  As stated previously, color blended compositing only works correctly if
  the images are rendered with a clear black background. Otherwise the
  background color will be added multiple times by each process that
  contains geometry in that pixel. If the
  \CEnum{ICET\_CORRECT\_COLORED\_BACKGROUND} feature is enabled, this
  background color is blended back into the final composited image.


  \mansection{Return Value}

  On each \index{display~process}display process (as defined by
  \CFunc{icetAddTile}), \CFunc{icetCompositeImage} returns the fully
  composited image in an \CType{IceTImage} object. The contents of the
  image are undefined for any non-display process.

  If the \CEnum{ICET\_COMPOSITE\_ONE\_BUFFER} option is on and both a color
  and depth buffer is specified with \icetSetColorFormat and
  \icetSetDepthFormat, then the returned image might be missing the depth
  buffer.  The rational behind this option is that often both the color and
  depth buffer is necessary in order to composite the color buffer, but the
  composited depth buffer is not needed.  In this case, the compositing
  might save some time by not transferring depth information at the latter
  stage of compositing.

  The returned image uses memory buffers that will be reclaimed the next
  time \IceT renders or composites a frame.  Do not use this image after
  the next call to \CFunc{icetCompositeImage} (unless you have changed the
  \IceT context).


  \mansection{Errors}

  \begin{Description}[ICET\_INVALID\_OPERATION]
  \item[\CErrorEnum{ICET\_INVALID\_VALUE}]
    An argument is set to \textC{NULL} where data is required.
  \item[\CErrorEnum{ICET\_OUT\_OF\_MEMORY}]
    Not enough memory left to hold intermittent frame buffers and other
    temporary data.
  \end{Description}

  \CFunc{icetDrawFrame} may also indirectly raise an error if there is an
  issue with the strategy or callback.


  \mansection{Warnings}

  \begin{Description}[ICET\_INVALID\_OPERATION]
  \item[\CErrorEnum{ICET\_INVALID\_VALUE}]
    An argument to \CFunc{icetCompositeImage} is inconsistent with the
    current \IceT state.
  \end{Description}


  \mansection{Bugs}

  The images provided must match the format expected by \IceT or else
  unpredictable behavior may occur. The images must also be carefully
  rendered to follow the provided viewport and projections. Images that a
  color blended must be rendered with a black background and rendered with
  the correct alpha value.

  If compositing with color blending on, the image returned may have a
  black background instead of the \CArg{background\_color} requested.  This
  can be corrected by blending the returned image over the desired
  background.  This will be done for you if the
  \CEnum{ICET\_CORRECT\_COLORED\_BACKGROUND} feature is enabled.


  \mansection{Copyright}
  Copyright \copyright 2014 Sandia Corporation

  @copyright@

  \mansection{See Also}

  \CFuncSeeAlso{icetAddTile},
  \CFuncSeeAlso{icetBoundingBox},
  \CFuncSeeAlso{icetBoundingVertices},
  \CFuncSeeAlso{icetDrawCallback},
  \CFuncSeeAlso{icetDrawFrame},
  \CFuncSeeAlso{icetSetColorFormat},
  \CFuncSeeAlso{icetSetDepthFormat},
  \CFuncSeeAlso{icetSingleImageStrategy},
  \CFuncSeeAlso{icetStrategy}

\end{Name}


% These are emacs settings that go at the end of the file.
% Local Variables:
% writestamp-format: "%B %e, %Y"
% writestamp-prefix: "^\\\\setDate{"
% writestamp-suffix: "}$"
% End:
