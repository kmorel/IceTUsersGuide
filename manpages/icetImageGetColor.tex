% -*- latex -*-


\setDate{July 28, 2010}
% The following must all be on one line for latex2man to work.
\begin{Name}{3}{icetImageGetColor}{Kenneth Moreland}{\IceT Reference}{\CFunc{icetImageGetColor}, \icetImageGetDepth -- retrieve pixel data buffer from image}

  \label{manpage:icetImageGetDepth}
  \index{icetImageGetDepth|(textbf}
  
  \mansection{Synopsis}

  \textC{\#include <IceT.h>}

  \begin{Table}{4}
    \textC{IceTUByte *}&\icetImageGetColorub&\textC{(}\quad\CType{IceTImage}&\CArg{image}\quad\textC{)}; \\
    \textC{IceTUInt *}&\icetImageGetColorui&\textC{(}\quad\CType{IceTImage}&\CArg{image}\quad\textC{)}; \\
    \textC{IceTFloat *}&\icetImageGetColorf&\textC{(}\quad\CType{IceTImage}&\CArg{image}\quad\textC{)}; \\
    \\
    \textC{IceTFloat *}&\icetImageGetDepthf&\textC{(}\quad\CType{IceTImage}&\CArg{image}\quad\textC{)};
  \end{Table}

  
  \mansection{Description}

  The \icetImageGetColor suite of functions retrieve color data from images
  and the \icetImageGetDepth functions retrieve depth data from images.
  Each function returns a pointer to an internal buffer within the image.
  Writing to this data changes the data within the image object itself.
  Use the \icetImageGetColor and \icetImageGetDepth functions from within
  drawing callbacks to pass image data back to \IceT.

  The pixel data is always tightly packed in horizontal major order.  Color
  data that comprises tuples such as RGBA have the components for each
  pixel packed together in that order.  The first entry in the array
  corresponds to the pixel in the lower left corner of the image.  The next
  entry is immediately to the right of the first pixel, and so on.  The
  dimensions of the array can be retrieved with the \icetImageGetWidth and
  \icetImageGetHeight functions.

  Each of these functions returns a typed version of the image data array.
  They can only succeed if the type the request matches the internal type
  of the array.  It is an error, for example, to request unsigned byte
  color data when the image stores images as floating point colors.  You
  can use the \icetImageGetColorFormat and \icetImageGetDepthFormat to
  retrieve the format for the internal data storage (which also implies the
  base data type).  You can also use the \icetImageCopyColor and
  \icetImageCopyDepth functions to convert the image data to whatever
  format you like.

  Use \icetImageGetColorub to retrieve an array of 8-bit unsigned bytes.
  Using this function is only valid if the color format is
  \CEnum{ICET\_IMAGE\_COLOR\_RGBA\_UBYTE}.

  Use \icetImageGetColorui to retrieve an array of 32-bit unsigned
  integers.  Using this function is only valid if the color format is
  \CEnum{ICET\_IMAGE\_COLOR\_RGBA\_UBYTE}.  In this case, each 32-bit
  integer represents all four RGBA channels.  Accessing each pixel's color
  values as a single 32-bit integer is often faster than accessing it as 4
  independent 8-bit integers as most modern architectures can access 32-bit
  memory boundaries faster than independent 8-bit boundaries.

  Use \icetImageGetColorf to retrieve an array of floating point color
  values.  Using this function is only valid if the color format is
  \CEnum{ICET\_IMAGE\_COLOR\_RGBA\_FLOAT}.

  Use \icetImageGetDepthf to retrieve an array of floating point depth
  values.  Using this function is only valid if the depth format is
  \CEnum{ICET\_IMAGE\_DEPTH\_FLOAT}.


  \mansection{Return Value}

  Returns an appropriately typed array pointing to the internal color or
  depth values stored in the image object.  If there is an error,
  \textC{NULL} is returned.

  The memory returned should not be freed.  It is managed internally by
  \IceT.

  \mansection{Errors}

  \begin{Description}[ICET\_INVALID\_OPERATION]
  \item[\CErrorEnum{ICET\_INVALID\_OPERATION}]
    The internal color or depth format is incompatible with the type of
    array the function retrieves.
  \end{Description}


  \mansection{Warnings}

  None.


  \mansection{Bugs}

  None known.


  \mansection{Notes}

  There is no mechanism to automatically determine the data type from the
  color or depth format enumeration (returned from \icetImageGetColorFormat
  or \icetImageGetDepthFormat).  Instead, you must code internal logic to
  use an array of the appropriate type.  The reasoning behind this decision
  is that the format encodes the data layout in addition to the data type,
  and your code most understand the basic semantics of the data to do
  anything worthwhile with it.  If you want to write code that is
  indifferent to the underlying format of the image, use the
  \CFunc{icetImageCopyColor} and \CFunc{icetImageCopyDepth} functions to
  copy the data to a known format.


  \mansection{Copyright}
  Copyright \copyright 2010 Sandia Corporation

  Under the terms of Contract DE-AC04-94AL85000, there is a non-exclusive
  license for use of this work by or on behalf of the U.S. Government.
  Redistribution and use in source and binary forms, with or without
  modification, are permitted provided that this Notice and any statement
  of authorship are reproduced on all copies.


  \mansection{See Also}

  \CFuncSeeAlso{icetImageCopyColor},
  \CFuncSeeAlso{icetImageCopyDepth},
  \CFuncSeeAlso{icetImageGetColorFormat},
  \CFuncSeeAlso{icetImageGetDepthFormat}

  \index{icetImageGetDepth|)textbf}

\end{Name}


% These are emacs settings that go at the end of the file.
% Local Variables:
% writestamp-format: "%B %e, %Y"
% writestamp-prefix: "^\\\\setDate{"
% writestamp-suffix: "}$"
% End:
