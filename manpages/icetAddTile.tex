% -*- latex -*-


\setDate{August 10, 2010}
% The following must all be on one line for latex2man to work.
\begin{Name}{3}{icetAddTile}{Kenneth Moreland}{\IceT Reference}{\CFunc{icetAddTile} -- add a tile to the logical display.}

  \mansection{Synopsis}

  \textC{\#include <IceT.h>}

  \begin{Table}{3}
    \textC{int }\CFunc{icetAddTile}\textC{(}&\textC{IceTInt}&\CArg{x}\textC{,} \\
      &\textC{IceTInt}&\CArg{y}\textC{,} \\
      &\textC{IceTSizeType}&\CArg{width}\textC{,} \\
      &\textC{IceTSizeType}&\CArg{height}\textC{,} \\
      &\textC{int}&\CArg{display\_rank}\quad\textC{);}
  \end{Table}

  
  \mansection{Description}

  Adds a tile to the tiled display.  Every process, whether actually
  displaying a tile or not, must declare the tiles in the display and which
  processes drive them with \CFunc{icetResetTiles} and \CFunc{icetAddTile}.
  Thus, each process calls \CFunc{icetAddTile} once for each tile in the
  display, and all processes must declare them in the same order.

  The parameters \CArg{x}, \CArg{y}, \CArg{width}, and \CArg{height} define
  the tile's viewport in the logical global display much in the same way
  \CFuncExternal{glViewport} declares a region in a physical display in
  \OpenGL.  \IceT places no limits on the extents of the logical global
  display.  That is, there are no limits on the values of \CArg{x} and
  \CArg{y}.  They can extend as far as they want in both the positive and
  negative directions.

  \IceT will project its images onto the region of the logical global
  display that just covers all of the tiles.  Therefore, shifting all the
  tiles in the logical global display by the same amount will have no real
  overall effect.

  The \CArg{display\_rank} parameter identifies the rank of the process
  that will be displaying the given tile.  It is assumed that the output of
  the rendering window of the given process is projected onto the space in
  a tiled display given by \CArg{x}, \CArg{y}, \CArg{width}, and
  \CArg{height}.  Each tile must have a valid rank (between $0$ and
  $\CEnum{ICET\_NUM\_PROCESSES} - 1$).  Furthermore, no process may be
  displaying more than one tile.


  \mansection{Return Value}

  Returns the index of the tile created or $-1$ if the tile could not be
  created.


  \mansection{Errors}

  \begin{Description}[ICET\_INVALID\_OPERATION]
  \item[\CErrorEnum{ICET\_INVALID\_VALUE}]
    Raised if \CArg{display\_rank} is not a valid process rank or
    \CArg{display\_rank} is already assigned to another tile.  If this
    error is raised, nothing is done and -1 is returned.
  \end{Description}


  \mansection{Warnings}

  None.


  \mansection{Bugs}

  All processes must specify the same tiles in the same order.  \IceT
  will assume this even though it is not explicitly detected or enforced.


  \mansection{Copyright}
  Copyright \copyright 2003 Sandia Corporation

  @copyright@


  \mansection{See Also}

  \CFuncSeeAlso{icetResetTiles},
  \CFuncSeeAlso{icetPhysicalRenderSize}

\end{Name}


% These are emacs settings that go at the end of the file.
% Local Variables:
% writestamp-format: "%B %e, %Y"
% writestamp-prefix: "^\\\\setDate{"
% writestamp-suffix: "}$"
% End:
