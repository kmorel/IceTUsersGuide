% -*- latex -*-


\setDate{August  9, 2010}
% The following must all be on one line for latex2man to work.
\begin{Name}{3}{icetSetContext}{Kenneth Moreland}{\IceT Reference}{\CFunc{icetSetContext} -- changes the current context.}

  \mansection{Synopsis}

  \textC{\#include <IceT.h>}

  \begin{Table}{3}
    \textC{void }\CFunc{icetSetContext}\textC{(}&\CType{IceTContext}&\CArg{context}\quad\textC{);}
  \end{Table}

  
  \mansection{Description}

  The \CFunc{icetSetContext} function sets the \IceT state machine to work
  with the context defined by \CArg{context} and the state associated with
  it.  Further calls to \IceT functions will operate based on the state
  encapsulated in \CArg{context}.  Changing the state of the context is a
  fast operation.


  \mansection{Errors}

  \begin{Description}[ICET\_INVALID\_OPERATION]
  \item[\CErrorEnum{ICET\_INVALID\_VALUE}]
    \CArg{context} is not valid.
  \end{Description}


  \mansection{Warnings}

  None.


  \mansection{Bugs}

  None known.


  \mansection{Notes}

  The behavior of \CFunc{icetSetContext} is somewhat indeterminate if
  \CArg{context} is not valid.  Usually, an
  \CErrorEnum{ICET\_INVALID\_VALUE} error will be raised, but it is
  possible that the context will be set to some other context.


  \mansection{Copyright}
  Copyright \copyright 2003 Sandia Corporation

  @copyright@

  \mansection{See Also}

  \CFuncSeeAlso{icetGetContext}, \CFuncSeeAlso{icetCreateContext},
  \CFuncSeeAlso{icetCopyState}

\end{Name}


% These are emacs settings that go at the end of the file.
% Local Variables:
% writestamp-format: "%B %e, %Y"
% writestamp-prefix: "^\\\\setDate{"
% writestamp-suffix: "}$"
% End:
