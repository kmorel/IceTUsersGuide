% -*- latex -*-


\setDate{August  9, 2010}
% The following must all be on one line for latex2man to work.
\begin{Name}{3}{icetWallTime}{Kenneth Moreland}{\IceT Reference}{\CFunc{icetWallTime} -- timer function}

  \mansection{Synopsis}

  \textC{\#include <IceT.h>}

  \begin{Table}{3}
    \textC{IceTDouble }\CFunc{icetWallTime}\textC{(}&\textC{void}&\textC{)}
  \end{Table}

  
  \mansection{Description}

  Retrieves the current time, in seconds.  The returned values of
  \CFunc{icetWallTime} are only valid in relation to each other.  That is,
  the time may or may not have anything to do with the current date or
  time.  However, the difference of values between two calls to
  \CFunc{icetWallTime} is the elapsed time in seconds between the two
  calls.  Thus, \CFunc{icetWallTime} is handy for determining the running
  time of various subprocesses.  \CFunc{icetWallTime} is used internally
  for determining the values for the state variables
  \CEnum{ICET\_BUFFER\_READ\_TIME}, \CEnum{ICET\_BUFFER\_WRITE\_TIME},
  \CEnum{ICET\_COMPARE\_TIME}, \CEnum{ICET\_COMPOSITE\_TIME},
  \CEnum{ICET\_COMPRESS\_TIME}, \CEnum{ICET\_RENDER\_TIME}, and
  \CEnum{ICET\_TOTAL\_DRAW\_TIME}.


  \mansection{Return Value}

  The current time, in seconds.


  \mansection{Errors}

  None.


  \mansection{Warnings}

  None.


  \mansection{Bugs}

  None known.


  \mansection{Copyright}
  Copyright \copyright 2003 Sandia Corporation

  @copyright@

  \mansection{See Also}

  \CFuncSeeAlso{icetGet}

\end{Name}


% These are emacs settings that go at the end of the file.
% Local Variables:
% writestamp-format: "%B %e, %Y"
% writestamp-prefix: "^\\\\setDate{"
% writestamp-suffix: "}$"
% End:
