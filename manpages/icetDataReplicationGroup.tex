% -*- latex -*-


\setDate{September 20, 2010}
% The following must all be on one line for latex2man to work.
\begin{Name}{3}{icetDataReplicationGroup}{Kenneth Moreland}{\IceT Reference}{\CFunc{icetDataReplicationGroup} -- define data replication.}

  \mansection{Synopsis}

  \textC{\#include <IceT.h>}

  \begin{Table}{3}
    \textC{void }\CFunc{icetDataReplicationGroup}\textC{(}&\textC{IceTInt}&\CArg{size}\textC{,} \\
      &\textC{const IceTInt *}&\CArg{processes}\quad\textC{);}
  \end{Table}

  
  \mansection{Description}

  \IceT has the ability to take advantage of geometric data that is
  replicated among processes.  If a group of processes share the same
  geometry data, then \IceT will split the region of the display that the
  data projects onto among the processes, thereby reducing the total amount
  of image composition work that needs to be done.

  Each group can be declared by calling \CFunc{icetDataReplicationGroup}
  and defining the group of processes that share the geometry with the
  local process.  \CArg{size} indicates how many processes belong to the
  group, and \CArg{processes} is an array of ids of processes that belong
  to the group.  Each process that belongs to a particular group must call
  \CFunc{icetDataReplicationGroup} with the exact same list of processes in
  the same order.

  You can alternately use \CFunc{icetDataReplicationGroupColor} to select
  data replication groups.

  By default, each process belongs to a group of size one containing just
  the local processes (i.e. there is no data replication).


  \mansection{Errors}

  \begin{Description}[ICET\_INVALID\_OPERATION]
  \item[\CErrorEnum{ICET\_INVALID\_VALUE}]
    \CArg{processes} does not contain the local process rank.
  \end{Description}


  \mansection{Warnings}

  None.


  \mansection{Bugs}

  \IceT assumes that \CFunc{icetDataReplicationGroup} is called with the
  exact same parameters on all processes belonging to a given group.
  Likewise, \IceT also assumes that all processes have called
  \CFunc{icetBoundingVertices} or \CFunc{icetBoundingBox} with the exact
  same parameters on all processes belonging to a given group.  These
  requirements are not strictly enforced, but failing to do so may cause
  some of the geometry to not be rendered.


  \mansection{Copyright}
  Copyright \copyright 2003 Sandia Corporation

  @copyright@

  \mansection{See Also}

  \CFuncSeeAlso{icetDataReplicationGroupColor},
  \CFuncSeeAlso{icetBoundingVertices},
  \CFuncSeeAlso{icetBoundingBox}

\end{Name}


% These are emacs settings that go at the end of the file.
% Local Variables:
% writestamp-format: "%B %e, %Y"
% writestamp-prefix: "^\\\\setDate{"
% writestamp-suffix: "}$"
% End:
