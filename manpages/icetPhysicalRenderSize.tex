% -*- latex -*-


\setDate{August  9, 2010}
% The following must all be on one line for latex2man to work.
\begin{Name}{3}{icetPhysicalRenderSize}{Kenneth Moreland}{\IceT Reference}{\CFunc{icetPhysicalRenderSize} -- set the size of images that are rendered}

  \mansection{Synopsis}

  \textC{\#include <IceT.h>}

  \begin{Table}{3}
    \textC{void }\CFunc{icetPhysicalRenderSize}\textC{(}&\textC{IceTInt}&\CArg{width}\textC{,} \\
      &\textC{IceTInt}&\CArg{height}\quad\textC{);}
  \end{Table}

  
  \mansection{Description}

  Specify the size of images that are rendered with
  \CFunc{icetPhysicalRenderSize}.  This is the size of image that is
  expected to be rendered by the draw callback (specified with
  \CFunc{icetDrawCallback}).  The width and height are captured in the
  \CEnum{ICET\_PHYSICAL\_RENDER\_WIDTH} and
  \CEnum{ICET\_PHYSICAL\_RENDER\_HEIGHT} state variables.

  The size of images that are rendered do not have to be the same size as
  the tile they are rendering so long as they are not smaller in any
  dimension.  In fact, when rendering multiple tiles it can often save time
  to render larger images.  Nevertheless, by default the physical render
  size is set to the size of the tiles in \CFunc{icetAddTile} because this
  is the most common use case.

  When using the \OpenGL rendering layer, the physical rendering size is
  overridden to the size of the \OpenGL viewport in each call to
  \CFunc{icetGLDrawFrame}.


  \mansection{Errors}

  None.


  \mansection{Warnings}

  \begin{Description}[ICET\_INVALID\_OPERATION]
  \item[\CErrorEnum{ICET\_INVALID\_VALUE}]
    The \CArg{width} or \CArg{height} specified is smaller than that for
    the largest tile.
  \end{Description}


  \mansection{Bugs}

  None known.


  \mansection{Copyright}
  Copyright \copyright 2010 Sandia Corporation

  @copyright@

  \mansection{See Also}

  \CFuncSeeAlso{icetAddTile}

\end{Name}


% These are emacs settings that go at the end of the file.
% Local Variables:
% writestamp-format: "%B %e, %Y"
% writestamp-prefix: "^\\\\setDate{"
% writestamp-suffix: "}$"
% End:
