% -*- latex -*-


\setDate{August  9, 2010}
% The following must all be on one line for latex2man to work.
\begin{Name}{3}{icetDestroyContext}{Kenneth Moreland}{\IceT Reference}{\CFunc{icetDestroyContext} -- delete a context.}

  \mansection{Synopsis}

  \textC{\#include <IceT.h>}

  \begin{Table}{3}
    \textC{void }\CFunc{icetDestroyContext}\textC{(}&\CType{IceTContext}&\CArg{context}\quad\textC{;}
  \end{Table}

  
  \mansection{Description}

  Frees the memory required to hold the state of \CArg{context} and removes
  \CArg{context} from existence.


  \mansection{Errors}

  None.


  \mansection{Warnings}

  None.


  \mansection{Bugs}

  \CFunc{icetDestroyContext} will happily delete the current context for
  you, but subsequent calls to most other \IceT functions will probably
  result in seg-faults unless you make another context current with
  \CFunc{icetCreateContext} or \CFunc{icetSetContext}.  The most notable
  execptions are the functions with names matching \textCF{icet*Context},
  which will work correctly without a proper current context.


  \mansection{Notes}

  Behavior is undefined if \CArg{context} has never been created or has
  already been destroyed.


  \mansection{Copyright}
  Copyright \copyright 2003 Sandia Corporation

  @copyright@

  \mansection{See Also}

  \CFuncSeeAlso{icetCreateContext}

\end{Name}


% These are emacs settings that go at the end of the file.
% Local Variables:
% writestamp-format: "%B %e, %Y"
% writestamp-prefix: "^\\\\setDate{"
% writestamp-suffix: "}$"
% End:
