% -*- latex -*-


\setDate{August  9, 2010}
% The following must all be on one line for latex2man to work.
\begin{Name}{3}{icetDiagnostics}{Kenneth Moreland}{\IceT Reference}{\CFunc{icetDiagnostics} -- change diagnostic reporting level.}

  \mansection{Synopsis}

  \textC{\#include <IceT.h>}

  \begin{Table}{3}
    \textC{void }\CFunc{icetDiagnostics}\textC{(}&\textC{IceTBitField}&\CArg{mask}\quad\textC{);}
  \end{Table}

  
  \mansection{Description}

  Sets what diagnostic message are printed to standard output.  The
  messages to be printed out are defined by \CArg{mask}.  \CArg{mask}
  consists of flags that are OR-ed together.  The valid flags are:
  \begin{Description}[xxxxxxxx]
  \item[\CEnum{ICET\_DIAG\_OFF}]
    A zero flag used to indicate that no diagnostic messages are desired.
  \item[\CEnum{ICET\_DIAG\_ERRORS}]
    Print messages associated with anomalous conditions.
  \item[\CEnum{ICET\_DIAG\_WARNINGS}]
    Print messages associated with conditions that are unexpected or may lead
    to errors.  Implicitly turns on \CEnum{ICET\_DIAG\_ERRORS}.
  \item[\CEnum{ICET\_DIAG\_DEBUG}]
    Print frequent messages concerning the status of \IceT.  Implicitly
    turns on \CEnum{ICET\_DIAG\_ERRORS} and \CEnum{ICET\_DIAG\_WARNINGS}.
  \item[\CEnum{ICET\_DIAG\_ROOT\_NODE}]
    Print messages only on the node with a process rank of 0.  This is the
    default if neither \CEnum{ICET\_DIAG\_ROOT\_NODE} nor
    \CEnum{ICET\_DIAG\_ALL\_NODES} is set.
  \item[\CEnum{ICET\_DIAG\_ALL\_NODES}]
    Print messages all every nodes.
  \item[\CEnum{ICET\_DIAG\_FULL}]
    Turn on all diagnostic messages on all nodes.
  \end{Description}

  The default flags are \CEnum{ICET\_DIAG\_ALL\_NODES} $|$
  \CEnum{ICET\_DIAG\_WARNINGS}.


  \mansection{Errors}

  None.


  \mansection{Warnings}

  None.


  \mansection{Bugs}

  None known.


  \mansection{Copyright}
  Copyright \copyright 2003 Sandia Corporation

  @copyright@

  \mansection{See Also}

  \CFuncSeeAlso{icetGetError}

\end{Name}


% These are emacs settings that go at the end of the file.
% Local Variables:
% writestamp-format: "%B %e, %Y"
% writestamp-prefix: "^\\\\setDate{"
% writestamp-suffix: "}$"
% End:
