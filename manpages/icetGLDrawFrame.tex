% -*- latex -*-


\setDate{September 22, 2014}
% The following must all be on one line for latex2man to work.
\begin{Name}{3}{icetGLDrawFrame}{Kenneth Moreland}{\IceT Reference}{\CFunc{icetGLDrawFrame} -- renders and composites a frame using \OpenGL}

  \mansection{Synopsis}

  \textC{\#include <IceTGL.h>}

  \begin{Table}{3}
    \textC{void}\quad\CFunc{icetGLDrawFrame}\textC{( void );}
  \end{Table}

  
  \mansection{Description}

  Initiates a frame draw using the current \OpenGL transformation matrices
  (modelview and projection).

  Before \IceT may render an image, the tiled display needs to be defined
  (using \CFunc{icetAddTile}), the drawing function needs to be set (using
  \CFunc{icetGLDrawCallback}), and composite strategy must be set (using
  \CFunc{icetStrategy}).  The single image sub-strategy may also optionally
  be set (using \CFunc{icetSingleImageStrategy}).

  All processes in the current \IceT context must call
  \CFunc{icetGLDrawFrame} for it to complete.

  The \OpenGL projection matrix and modelview matrix must be set using
  \CFuncExternal{glLoadMatrix} or a number of other functions.  Likewise,
  the \OpenGL background color must be set with
  \CFuncExternal{glClearColor}.  \IceT will use the matrices to determine
  what pixels need to be rendered and composited.  \IceT will also modify
  the projection matrix to project onto (or in between) tiles.

  If \CEnum{ICET\_GL\_DISPLAY} is enabled, then the fully composited image
  is written back to the \OpenGL framebuffer for display.  It is the
  application's responsibility to synchronize the processes and swap front
  and back buffers.  If the \OpenGL background color is set to something
  other than black, \CEnum{ICET\_GL\_DISPLAY\_COLORED\_BACKGROUND} should
  also be enabled.  Displaying with
  \CEnum{ICET\_GL\_DISPLAY\_COLORED\_BACKGROUND} disabled may be slightly
  faster (depending on graphics hardware) but can result in black
  rectangles in the background.  If \CEnum{ICET\_GL\_DISPLAY\_INFLATE} is
  enabled and the size of the renderable window (determined by the current
  \OpenGL viewport) is greater than that of the tile being displayed, then
  the image will first be ``inflated'' to the size of the actual display.
  This inflation will be assisted by texture hardware if
  \CEnum{ICET\_GL\_DISPLAY\_INFLATE\_WITH\_HARDWARE} is on.  If
  \CEnum{ICET\_GL\_DISPLAY\_INFLATE} is disabled, the image is drawn at its
  original resolution at the lower left corner of the display.

  The image remaining in the frame buffer is undefined if
  \CEnum{ICET\_GL\_DISPLAY} is disabled or the process is not displaying a
  tile.


  \mansection{Errors}

  \begin{Description}[ICET\_INVALID\_OPERATION]
  \item[\CErrorEnum{ICET\_INVALID\_OPERATION}]
    Raised if the \CFunc{icetGLInitialize} has not been called or if the
    drawing callback has not been set.  Also can be raised if
    \CFunc{icetDrawFrame} is called recursively, probably from within the
    drawing callback.
  \item[\CErrorEnum{ICET\_OUT\_OF\_MEMORY}]
    Not enough memory left to hold intermittent frame buffers and other
    temporary data.
  \end{Description}

  \CFunc{icetGLDrawFrame} may also indirectly raise an error if there is an
  issue with the strategy or callback.


  \mansection{Warnings}

  None.


  \mansection{Bugs}

  If compositing with color blending on, the image returned may have a
  black background instead of the \CArg{background\_color} requested.  This
  can be corrected by blending the returned image over the desired
  background.  This will be done for you if the
  \CEnum{ICET\_CORRECT\_COLORED\_BACKGROUND} is on.


  \mansection{Copyright}
  Copyright \copyright 2003 Sandia Corporation

  @copyright@

  \mansection{See Also}

  \CFuncSeeAlso{icetAddTile},
  \CFuncSeeAlso{icetBoundingBox},
  \CFuncSeeAlso{icetBoundingVertices},
  \CFuncSeeAlso{icetDrawFrame},
  \CFuncSeeAlso{icetGLDrawCallback},
  \CFuncSeeAlso{icetSingleImageStrategy},
  \CFuncSeeAlso{icetStrategy}

\end{Name}


% These are emacs settings that go at the end of the file.
% Local Variables:
% writestamp-format: "%B %e, %Y"
% writestamp-prefix: "^\\\\setDate{"
% writestamp-suffix: "}$"
% End:
