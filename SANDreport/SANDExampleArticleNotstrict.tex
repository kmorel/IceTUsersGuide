%
% $Id$
%
% This is an example LaTeX file which uses the SANDreport class file.
% It shows how a SAND report should be formatted, what sections and
% elements it should contain, and how to use the SANDreport class.
% It uses the LaTeX article class, but not the strict option.
% It uses jpeg logos and files to show how pdflatex can be used
%
% Get the latest version of the class file and more at
%    http://www.cs.sandia.gov/~rolf/SANDreport
%
% This file and the SANDreport.cls file are based on information
% contained in "Guide to Preparing {SAND} Reports", Sand98-0730, edited
% by Tamara K. Locke, and the newer "Guide to Preparing SAND Reports and
% Other Communication Products", SAND2002-2068P.
% Please send corrections and suggestions for improvements to
% Rolf Riesen, Org. 9223, MS 1110, rolf@cs.sandia.gov
%
\documentclass[pdf,ps2pdf,12pt]{SANDreport}
\usepackage{pslatex}
\usepackage{mathptmx}	% Use the Postscript Times font
\usepackage[FIGBOTCAP,normal,bf,tight]{subfigure}


% If you want to relax some of the SAND98-0730 requirements, use the "relax"
% option. It adds spaces and boldface in the table of contents, and does not
% force the page layout sizes.
% e.g. \documentclass[relax,12pt]{SANDreport}
%
% You can also use the "strict" option, which applies even more of the
% SAND98-0730 guidelines. It gets rid of section numbers which are often
% useful; e.g. \documentclass[strict]{SANDreport}



% ---------------------------------------------------------------------------- %
%
% Set the title, author, and date
%
    \title{On the Use of Magic at National Laboratories (Article style, not strict,
	using PDF logos and JPG graphics)}

    \author{Harry E. Potter \\
	  Special Projects Department \\
	  Sandia National Laboratories\\
	  P.O. Box 5800\\
	  Albuquerque, NM 87185-9999 \\
	  hpotter@sandia.gov \\
	  \\
	  \and
	  Hermoine S. Granger \\
	  Hogwarts \\
	  Hogsmeade, North of London, Great Britain\\
	  granger@hogwarts.edu
	 }

    % There is a "Printed" date on the title page of a SAND report, so
    % the generic \date should generally be empty.
    \date{}


% ---------------------------------------------------------------------------- %
% Set some things we need for SAND reports. These are mandatory
%
\SANDnum{SAND2002-xxxx}
\SANDprintDate{June 2002}
\SANDauthor{Harry E. Potter, Hermoine S. Granger}


% ---------------------------------------------------------------------------- %
% Include the markings required for your SAND report. The default is "Unlimited
% Release". You may have to edit the file included here, or create your own
% (see the examples provided).
%
% \include{MarkUR} % Not needed for unlimted release reports



% ---------------------------------------------------------------------------- %
% The following definition does not have a default value and will not
% print anything, if not defined
%
\SANDsupersed{SAND1901-0001}{January 1901}


% ---------------------------------------------------------------------------- %
%
% Start the document
%
\begin{document}
    \maketitle

    % ------------------------------------------------------------------------ %
    % An Abstract is required for SAND reports
    %
    \begin{abstract}
	It would appear that many projects carried out at the
	national DOE laboratories could benefit from some magic. This
	report examines the requirements for the use of magic and
	gives a few specific examples on how magic can be used to
	advance the state of the art.
    \end{abstract}


    % ------------------------------------------------------------------------ %
    % An Acknowledgement section is optional but important, if someone made
    % contributions or helped beyond the normal part of a work assignment.
    % Use \section* since we don't want it in the table of context
    %
    \clearpage
    \section*{Acknowledgment}
	Thanks to Ron Weasly for valuable discussions and helping
	us finding new uses of magic.

	The format of this report is based on information found
	in~\cite{Sand98-0730}.


    % ------------------------------------------------------------------------ %
    % The table of contents and list of figures and tables
    % Comment out \listoffigures and \listoftables if there are no
    % figures or tables. Make sure this starts on an odd numbered page
    %
    \cleardoublepage		% TOC needs to start on an odd page
    \tableofcontents
    \listoffigures
    \listoftables


    % ---------------------------------------------------------------------- %
    % An optional preface or Foreword
    \clearpage
    \section*{Preface}
    \addcontentsline{toc}{section}{Preface}
	Although muggles usually have only limited experience with
	magic, and many even dispute its existence, it is worthwhile
	to be open minded and explore the possibilities.


    % ---------------------------------------------------------------------- %
    % An optional executive summary
    \clearpage
    \section*{Summary}
    \addcontentsline{toc}{section}{Summary}
	Once a certain level of mistrust and skepticism has
	been overcome, magic finds many uses in todays science
	and engineering. In this report we explain some of the
	fundamental spells and instruments of magic and wizardry. We
	then conclude with a few examples on how they can be used
	in daily activities at national Laboratories.


    % ---------------------------------------------------------------------- %
    % An optional glossary. We don't want it to be numbered
    \clearpage
    \section*{Nomenclature}
    \addcontentsline{toc}{section}{Nomenclature}
    \begin{description}
	\item[alohomoral]
	    spell to open locked doors and containers
	\item[leviosa]
	    spell to levitate objects
	\item[remembrall]
	    device to alert you that you have forgotten something
	\item[wand]
	    device to execute spells
    \end{description}


    % ---------------------------------------------------------------------- %
    % This is where the body of the report begins; usually with an Introduction
    %
    \SANDmain		% Start the main part of the report

    \section{Introduction}
	\label{Intro}
	In~\cite{Potter} we have shown that magic has a use in
	muggle science. In this report we address the use of magic
	specifically to the science and engineering performed at the
	DOE national laboratories. Let us begin with a description
	of a couple of basic magical shapes. We use the circle and
	the square to make our point.

	\subsection{The Circle}
	    Figure~\ref{fig1} shows one of the basic magical shapes. It
	    appears in many spells and has many important properties.

	    \begin{figure}[ht]
		\centering
		\begin{picture}(50,50)(0,0)
		    \put(25,25){\circle{50}}
		    \put(25,25){\circle{1}}
		\end{picture}
		\caption[The circle]{A circle is one of basic magical shapes.
		    It has many important properties. Note, for example, that
		    every point along its circumference is at exactly the same
		    distance from the center. A truly magical property.}
		\label{fig1}
	    \end{figure}

	    Whole books have been written about the circle and
	    similar, lesser shapes. So, it would be presumptuous to
	    attempt to even list some of its properties and magical
	    uses. Simply admire the shape in Figure~\ref{fig1} and
	    its power will soon engulf you in ideas and beautiful
	    feelings of completeness.


	\subsubsection{Small Circles}
	    We also need an example of a subsubsection to make
	    sure tiles and table of content entries work for those.

	    And while we are at it, we might as well check paragraph
	    spacing. Although, we will do that again in the bla section.

	\subsection{The Square}
	    Figure~\ref{fig2} illustrates another famous magical shape.
	    We will show some of its uses in later chapters, but we
	    wanted to introduce it here because this is a good place
	    for another figure.

	    \begin{figure}[ht]
		\centering
		\begin{picture}(50,50)(0,0)
		    \put(0,0){\framebox(50,50){}}
		    \put(25,25){\circle{1}}
		\end{picture}
		\caption[The square]{A square is another of the basic magical
		    shapes. It is not quite as powerful as the circle. It
		    has some similarities (note that the four corners all have
		    the same distance to the center), and has many fine
		    uses in everyday magic.}
		\label{fig2}
	    \end{figure}

	\subsection{The Dot}
	    We also need a figure with a short caption. In order to do this,
	    we introduce the dot. Figure~\ref{fig3} shows the shape of a dot.

	    \begin{figure}[ht]
		\centering
		\begin{picture}(50,50)(0,0)
		    \put(25,25){\circle{1}}
		\end{picture}
		\caption[The dot]{A simple dot}
		\label{fig3}
	    \end{figure}

	\subsection{Tables and Such}
	    In order to test our class file, we also need to
	    have some tables. One should be enough for our purposes,
	    so here it is: Table~\ref{tab1}. On second thought, we
	    need another one to test the list of tables with multiple
	    entries. So, we introduce Table~\ref{tab2}.

	    \begin{table}[ht]
		\centering
		\caption[Magical shapes]{This superb table lists a few
		    of the more important magical shapes and some of
		    their properties. Be aware that this condensed list
		    can by no means describe all the properties or
		    shapes in use by modern magic.}
		\bigskip

		\begin{tabular}{|l|c|l|c|}
		    \hline \hline
		    Name  & Number of & Importance & Shape \\
		          & corners   &            &       \\
		    \hline
		    circle & 0        & high       & $\bigcirc$ \\
		    square & 4        & medium     & $\diamond$ \\
		    triangle & 3      & low        & $\triangle$ \\
		    \hline
		\end{tabular}
		\label{tab1}
	    \end{table}

	    \begin{table}[ht]
		\centering
		\caption{A magic square}
		\bigskip

		\begin{tabular}{|c|c|c|c|}
		    \hline
			1 & 15 & 14 & 4 \\ \hline
			12 & 6 & 7 & 9 \\ \hline
			8 & 10 & 11 & 5 \\ \hline
			13 & 3 & 2 & 16 \\ \hline
		\end{tabular}
		\label{tab2}
	    \end{table}


    \section{A Section With Subfigures}
	The subfigure package used to cause problems until James
	Gruetzner and Todd Pitts found out that the subfigure
	package uses {\tt addcontentsline}, which is redefined in
	the SANDreport class. The class uses the ifthen package,
	but it is not loaded at the time the subfigure package
	uses {\tt addcontentsline}.  The new code avoids using the
	ifthen package.  Have a look at Figure~\ref{fig:creatures}
	for an example.

	\begin{figure}[!btp]
	    \centering
	    \subfigure[A Hyppogriff]{
		\label{fig:sub:intro:creatures:hippogriff}
		\includegraphics[keepaspectratio=true, width=2.0in]{hippogriff.jpg}
	    }
	    \subfigure[A Dragon]{
		\label{fig:sub:intro:creatures:dragon}
		\includegraphics[keepaspectratio=true, width=2.0in]{dragon.jpg}
	    }
	    \caption{Magical Creatures.}
	    \label{fig:creatures}
	\end{figure}



    \section{A Long Section}
	We need a long section to test two-sided formatting. Therefore,
	we introduce the concept of {\em bla}. We will discover in a
	moment, that there are many bla's in this section. Without further
	ado, here they are.

	\newcommand{\myblaA}{bla bla bla bla bla bla bla bla bla bla }
	\newcommand{\myblaB}{\myblaA \myblaA \myblaA \myblaA \myblaA
	    \myblaA \myblaA \myblaA \myblaA \myblaA }

	\myblaB \myblaB

	\myblaB \myblaB \myblaB \myblaB

	\myblaB \myblaB \myblaB 

	\myblaB \myblaB \myblaB

	\myblaB \myblaB \myblaB \myblaB \myblaB

	\myblaB \myblaB

    \section{Conclusion}
	Of course, no report would be complete without some conclusions.
	This section is where they would go, if we had some.

    \nocite{*}


    % ---------------------------------------------------------------------- %
    % References
    %
    \clearpage
    % If hyperref is included, then \phantomsection is already defined.
    % If not, we need to define it.
    \providecommand*{\phantomsection}{}
    \phantomsection
    \addcontentsline{toc}{section}{References}
    \bibliographystyle{plain}
    \bibliography{SANDExample}


    % ---------------------------------------------------------------------- %
    %
    \appendix
    \section{Historical Perspective}
	This is an example of an appendix.

	If we follow~\cite{Sand98-0730} strictly, we would have to
	have a separate bibliography section for each appendix.
	The style file doesn't provide that, but it can be done
	using the {\tt bibunits} and {\tt chapterbib} packages.

	If there are many subsections in an appendix, it should also
	have its own table of contents. Again, the SAND report class
	file does not provide that functionality.

	\subsection{The Past a Long Time Ago}
	    This is where we talk about things so old nobody
	    can verify them. We are safe.

	\subsection{The Past More Recently}
	    Now we have to be a little bit more careful, since
	    records exist from that time, and some people still
	    alive actually lived back then.


    \section{Some Other Appendix}
	Just to show what a second Appendix would look like. It contains
	a table. Each appendix is supposed to be self-contained, so
	tables and figures are not supposed to show up in the main
	table of contents. There can be a separate table of contents
	for each appendix.

	\begin{table}[ht]
	    \centering
	    \caption{A small table}
	    \bigskip

	    \begin{tabular}{|c|c|}
		\hline
		    A & B  \\ \hline
		    C & D  \\ \hline
	    \end{tabular}
	    \label{tab3}
	\end{table}

	\begin{figure}[ht]
	    \centering
	    \begin{picture}(50,50)(0,0)
		\put(25,25){\circle{1}}
		\put(25,25){\circle{5}}
		\put(25,25){\circle{10}}
		\put(25,25){\circle{15}}
		\put(25,25){\circle{20}}
		\put(25,25){\circle{25}}
		\put(25,25){\circle{30}}
		\put(25,25){\circle{35}}
		\put(25,25){\circle{40}}
		\put(25,25){\circle{45}}
		\put(25,25){\circle{50}}
	    \end{picture}
	    \caption{Dizzy yet?}
	    \label{fig4}
	\end{figure}

    % \printindex

    \include{SANDdistribution}

\end{document}
