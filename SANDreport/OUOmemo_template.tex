%
% Version Apr 28, 2008
%
% This is a template to create a Sandia Memo using LaTeX; this version
% demonstrates how to mark a memo OUO.
%
% I believe the class file was started ~10 years ago in 1520 by Mark
% Blanford. It was rewritten in October of 2004 by a contractor (see
% sfonts.sty for contact information). Rich Field has made a few
% very minor changes since then.
%
\documentclass[pdf,ps2pdf,12pt]{smemo}
\usepackage{epsfig}
\usepackage[FIGBOTCAP,normal,bf,tight]{subfigure}
\usepackage{wrapfig}

% ---------------------------------------------------------------------- %
% Select PostScript fonts and mark it OUO if desired
%

    % If you want to use PostScript fonts, turn this on. sfonts.sty
    % may need to be configured for your system. See instruction in
    % the sfonts.sty file
    %
    % \usepackage{sfonts}

\input{markmemoOUO} % provides headers and footers for OUO documents

% ---------------------------------------------------------------------- %
% Start the document
%
\begin{document}
    \begin{memo}

    \thispagestyle{plain}  % needed to remove headers/footers from cover page
    % OUObox.tex: insert boxed OUO table on page 1

\begin{table}[!b]
\vskip50pt
\hskip2.5in
\renewcommand{\arraystretch}{0.9}
\begin{tabular}{|p{3.8in}|}
\hline
\footnotesize{\textbf{OFFICIAL USE ONLY}} \\
\scriptsize{May be exempt from public release under the Freedom of Information
  Act (5 U.S.C. 552), exemption number and category: 3. Statutory Exemption.}  \\
\scriptsize{Department of Energy review required before public release} \\
\scriptsize{Name/Org:  \underline{A. Dumbledore / 00001} Date:
  \underline{\today}}\\ 
\scriptsize{Guidance (if applicable):} \\
\hline
\end{tabular}
\end{table}

         % draws OUO box - edit with your name/Org

    % ---------------------------------------------------------------------- %
    % Header information
    %

    % If different from today's; or use \nodate to suppress date
    \date{}

    % if to one person, put their name here
    % e.g. \to{H. S. Granger, Org. 0001, MS-9999}
    % otherwise use \to{Distribution} and include distribution list at end
    \to{Distribution}

    % For running head, if heading on following pages is not Addressee
    %\headtext{}

    % Who is this memo from
    \from{J. E. Smith, Org. 0002, MS-0001}

    % Fill in a subject
    \subject{On the Use of Dry Erase Markers at National Laboratories (SAMPLE -
      CONTAINS NO OUO)}

    % ---------------------------------------------------------------------- %
    % Main text of memo begins here
    %
    It would appear that many projects carried out at the national DOE
    laboratories could benefit from some dry erase markers. This report examines the
    requirements for the use of dry erase markers and gives a few specific examples on
    how dry erase markers can be used to advance the state of the art.

    \section{Introduction}\label{Intro}
    In~\cite{Potter} we have shown that dry erase markers have
    a use in science. In this report we address the use of dry
    erase markers specifically to the science and engineering
    performed at the DOE national laboratories. Let us begin
    with a description of a couple of basic dry erase marker
    drawings. We use the circle and the square to make our point.

    \subsection{The Circle}
	    Figure~\ref{fig1} shows one of the basic shapes. It
	    appears in many dry erase drawings and has many important
	    properties.

	    \begin{figure}[ht]
		\centering
		\begin{picture}(50,50)(0,0)
		    \put(25,25){\circle{50}}
		    \put(25,25){\circle{1}}
		\end{picture}
		\caption[The circle]{A circle is one of the basic
		    shapes.  It has many important properties. Note,
		    for example, that every point along its
		    circumference is at exactly the same distance
		    from the center. A truly important property
		    for a cicrle.}
		\label{fig1}
	    \end{figure}

	    Whole books have been written about the circle and
	    similar, lesser shapes. So, it would be presumptuous to
	    attempt to even list some of its properties and uses
	    in dry erase marker drawings. Simply admire the shape
	    in Figure~\ref{fig1} and its power will soon engulf
	    you in ideas and beautiful feelings of completeness.



    \subsubsection{Small Circles}
    We also need an example of a subsubsection to make sure titles and
    table of content entries work for those.
    
    And while we are at it, we might as well check paragraph
    spacing. Although, we will do that again in Section~\ref{sec:long}.

    \subsection{The Square}
    Figure~\ref{fig2} illustrates another famous shape. We will
    show some of its uses in later chapters, but we wanted to introduce it
    here because this is a good place for another figure.
    
    \begin{figure}[ht]
      \centering
      \begin{picture}(50,50)(0,0)
        \put(0,0){\framebox(50,50){}}
        \put(25,25){\circle{1}}
      \end{picture}
      \caption[The square]{A square is another of the basic
        shapes. It is not quite as powerful as the circle. It
        has some similarities (note that the four corners all have
        the same distance to the center), and has many fine
        uses in everyday dry erase marker drawing.}
      \label{fig2}
    \end{figure}

    \subsection{The Dot}
    We also need a figure with a short caption. In order to do this,
    we introduce the dot. Figure~\ref{fig3} shows the shape of a dot.

    \begin{figure}[ht]
      \centering
      \begin{picture}(50,50)(0,0)
        \put(25,25){\circle{1}}
      \end{picture}
      \caption[The dot]{A simple dot}
      \label{fig3}
    \end{figure}

    \subsection{Tables and Such}
    In order to test our class file, we also need to have some tables. One
    should be enough for our purposes, so here it is: Table~\ref{tab1}. On
    second thought, we need another one to test the list of tables with
    multiple entries. So, we introduce Table~\ref{tab2}.
    
    \begin{table}[ht]
      \centering
      \caption[Shapes]{This superb table lists a few
        of the more important shapes and some of
        their properties. Be aware that this condensed list
        can by no means describe all the properties or
        shapes drawable by dry erase markers.}
      \bigskip
      
      \begin{tabular}{|l|c|l|c|}
        \hline \hline
        Name  & Number of & Importance & Shape \\
        & corners   &            &       \\
        \hline
        circle & 0        & high       & $\bigcirc$ \\
        square & 4        & medium     & $\diamond$ \\
        triangle & 3      & low        & $\triangle$ \\
        \hline
      \end{tabular}
      \label{tab1}
    \end{table}

    \begin{table}[ht]
      \centering
      \caption{A magic square}
      \bigskip
      
      \begin{tabular}{|c|c|c|c|}
        \hline
        1 & 15 & 14 & 4 \\ \hline
        12 & 6 & 7 & 9 \\ \hline
        8 & 10 & 11 & 5 \\ \hline
        13 & 3 & 2 & 16 \\ \hline
      \end{tabular}
      \label{tab2}
    \end{table}

    \section{A Section With Subfigures}
    The subfigure package used to cause problems until James Gruetzner and
    Todd Pitts found out that the subfigure package uses {\tt
      addcontentsline}, which is redefined in the SANDreport class. The
    class uses the ifthen package, but it is not loaded at the time the
    subfigure package uses {\tt addcontentsline}.  The new code avoids
    using the ifthen package.  Have a look at Figure~\ref{fig:creatures} 
    for an example.

    \begin{figure}[!btp]
      \centering
      \subfigure[A Hyppogriff]{
        \label{fig:sub:intro:creatures:hippogriff}
        \includegraphics[keepaspectratio=true, width=2.0in]{hippogriff}
      }
      \subfigure[A Dragon]{
        \label{fig:sub:intro:creatures:dragon}
        \includegraphics[keepaspectratio=true, width=2.0in]{dragon}
      }
      \caption{Creatures not drawn using dry erase markers.}
      \label{fig:creatures}
    \end{figure}
    
    \section{A Long Section}\label{sec:long}
    We need a long chapter to test full-page formatting. Therefore, we
    switch to the ancient language of Latin.

    \input{LoremIpsum}

    \section{Conclusions}
    Of course, no report would be complete without some conclusions.
    This section is where they would go, if we had some.

    %
    % Main text of memo ends here
    % ---------------------------------------------------------------------- %

    % ---------------------------------------------------------------------- %
    % Initials, enclosures, and keywords, if necessary
    % 
    % If initials are required; e.g., \initials{ H.P. }
    % \initials{}

    % If you have enclosures
    % \enc{}

    % Should you need keywords; e.g., \keywords{ dry erase markers }
    % \keywords{}
    
    % ---------------------------------------------------------------------- %
    % References
    % 
    \normalfont 
    \bibliographystyle{plain}
    \bibliography{smemoExample}

    % ---------------------------------------------------------------------- %
    % Appendices
    % 
    \clearpage
    \renewcommand{\thesection}{\Alph{section}}
    \setcounter{section}{0}

    \section{Historical Perspective}
    This is an example of an appendix.

    \subsection{The Past a Long Time Ago}
    This is where we talk about things so old nobody can verify them. We
    are safe. 

    \subsection{The Past More Recently}
    Now we have to be a little bit more careful, since records exist from
    that time, and some people still alive actually lived back then.

    \section{Some Other Appendix}
    Just to show what a second Appendix would look like. It contains
    a table. 

    \begin{table}[ht]
      \centering
      \caption{A small table}
      \bigskip
  
      \begin{tabular}{|c|c|}
        \hline
        A & B  \\ \hline
        C & D  \\ \hline
      \end{tabular}
      \label{tab3}
    \end{table}

    \begin{figure}[ht]
      \centering
      \begin{picture}(50,50)(0,0)
        \put(25,25){\circle{1}}
        \put(25,25){\circle{5}}
        \put(25,25){\circle{10}}
        \put(25,25){\circle{15}}
        \put(25,25){\circle{20}}
        \put(25,25){\circle{25}}
        \put(25,25){\circle{30}}
        \put(25,25){\circle{35}}
        \put(25,25){\circle{40}}
        \put(25,25){\circle{45}}
        \put(25,25){\circle{50}}
      \end{picture}
      \caption{Dizzy yet?}
      \label{fig4}
    \end{figure}

    \clearpage
    
    % ---------------------------------------------------------------------- %
    % Distribution list
    % 
    \begin{distribution}{External Distribution:}

      \normalfont
      \input{distE}

    \end{distribution}

    \begin{distribution}{Internal Distribution:}
      
      \normalfont
      \input{distI}

    \end{distribution}

% ---------------------------------------------------------------------- %
% End
% 

  \end{memo}
\end{document}
