%
% $Id: SANDExampleReportNotstrict.tex,v 1.26 2009/05/01 20:59:19 rolf Exp $
%
% This is an example LaTeX file which uses the SANDreport class file.
% It shows how a SAND report should be formatted, what sections and
% elements it should contain, and how to use the SANDreport class.
% It uses the LaTeX report class, but not the strict option.
%
% Get the latest version of the class file and more at
%    http://www.cs.sandia.gov/~rolf/SANDreport
%
% This file and the SANDreport.cls file are based on information
% contained in "Guide to Preparing {SAND} Reports", Sand98-0730, edited
% by Tamara K. Locke, and the newer "Guide to Preparing SAND Reports and
% Other Communication Products", SAND2002-2068P.
% Please send corrections and suggestions for improvements to
% Rolf Riesen, Org. 9223, MS 1110, rolf@cs.sandia.gov
%
\documentclass[pdf,ps2pdf,12pt,report,OUO]{SANDreport}
\usepackage{pslatex}
\usepackage{mathptmx}	% Use the Postscript Times font
\usepackage[FIGBOTCAP,normal,bf,tight]{subfigure}
\usepackage[dvips,light,first,bottomafter]{draftcopy}
\draftcopyName{Sample, contains no OUO}{70}

% If you want to relax some of the SAND98-0730 requirements, use the "relax"
% option. It adds spaces and boldface in the table of contents, and does not
% force the page layout sizes.
% e.g. \documentclass[relax,12pt]{SANDreport}
%
% You can also use the "strict" option, which applies even more of the
% SAND98-0730 guidelines. It gets rid of section numbers which are often
% useful; e.g. \documentclass[strict]{SANDreport}



% ---------------------------------------------------------------------------- %
%
% Set the title, author, and date
%
    \title{On the Use of Dry Erase Markers at National Laboratories (Report style, not strict)}

    \author{John E. Smith \\
	  Special Projects Department \\
	  Sandia National Laboratories\\
	  P.O. Box 5800\\
	  Albuquerque, NM 87185-9999 \\
	  root@sandia.gov \\
	  \\
	  \and
	  Jane S. Miller \\
	  Whatsamatter University \\
	  Northside, New York, NY\\
	  jsm@whatsamatteru.edu
	 }

    % There is a "Printed" date on the title page of a SAND report, so
    % the generic \date should generally be empty.
    \date{}


% ---------------------------------------------------------------------------- %
% Set some things we need for SAND reports. These are mandatory
%
\SANDnum{SAND2002-xxxx}
\SANDprintDate{June 2002}
\SANDauthor{Jonh E. Smith, Jane S. Miller}


% ---------------------------------------------------------------------------- %
% Include the markings required for your SAND report. The default is "Unlimited
% Release". You may have to edit the file included here, or create your own
% (see the examples provided).
%
% \include{MarkUR} % Not needed for unlimted release reports


% ---------------------------------------------------------------------------- %
% The following definition does not have a default value and will not
% print anything, if not defined
%
\SANDsupersed{SAND1901-0001}{January 1901}
%
% Include this file, if your SANDIA Report is Official Use Only.
% This is an example provided by Michael Kaneshige, and you'll have to
% change the wording and data according to your needs.
% See Appendix A in "Guide to Preparing SAND Reports and Other
% Communication Products". pages 55ff.
%
\SANDreleaseType{Official Use Only $\bullet$ Export Controlled Information}
\SANDmarkTopBottom{\CoverFont{b}{12}{10pt}OFFICIAL USE ONLY}

\SANDmarkCover{
  \framebox{
    \begin{minipage}{3.6in}
        \vspace{.1in}
	\begin{center}\textbf{OFFICIAL USE ONLY}\end{center}
	May be exempt from public release under the Freedom of 
	Information Act (5 U.S.C. 552), exemption number and 
        category: 3. Statutory Exemption.  \\
	\\
        Department of Energy review required before public release \\
	\\
	Name/Org:  \underline{Your Name~/~YourOrg} Date: \underline{January 12, 2007}\\ \\
	Guidance (if applicable): \\
    \end{minipage}
  } \\ \\

  \begin{minipage}{3.75in}
    \textbf{EXPORT CONTROLLED INFORMATION}\\ \\
    Further dissemination authorized to the Department of Energy and DOE contractors
    only; other requests shall be approved by the originating facility or higher DOE
    programmatic authority. \\
    \\
    Treat this material per Department of Sate (DOS) International Traffic and
    Arms Regulations, 22 CFR 120-130.  Information contained in this document
    is also subject to controls defined by the Department of Defense Directive
    5320.25.
  \end{minipage}
}



% ---------------------------------------------------------------------------- %
%
% Start the document
%
\begin{document}
    \maketitle

    % ------------------------------------------------------------------------ %
    % An Abstract is required for SAND reports
    %
    \begin{abstract}
	\input{CommonAbstract}
    \end{abstract}


    % ------------------------------------------------------------------------ %
    % An Acknowledgement section is optional but important, if someone made
    % contributions or helped beyond the normal part of a work assignment.
    % Use \section* since we don't want it in the table of context
    %
    \clearpage
    \chapter*{Acknowledgment}
	\input{CommonAck}


    % ------------------------------------------------------------------------ %
    % The table of contents and list of figures and tables
    % Comment out \listoffigures and \listoftables if there are no
    % figures or tables. Make sure this starts on an odd numbered page
    %
    \cleardoublepage		% TOC needs to start on an odd page
    \tableofcontents
    \listoffigures
    \listoftables


    % ---------------------------------------------------------------------- %
    % An optional preface or Foreword
    \clearpage
    \chapter*{Preface}
    \addcontentsline{toc}{chapter}{Preface}
	\input{CommonPreface}


    % ---------------------------------------------------------------------- %
    % An optional executive summary
    \clearpage
    \chapter*{Summary}
    \addcontentsline{toc}{chapter}{Summary}
	\input{CommonSummary}


    % ---------------------------------------------------------------------- %
    % An optional glossary. We don't want it to be numbered
    \clearpage
    \chapter*{Nomenclature}
    \addcontentsline{toc}{chapter}{Nomenclature}
    \begin{description}
	\item[dry spell]
	    using a dry erase marker to spell words
	\item[dry wall]
	    the writing on the wall
	\item[dry humor]
	    when people just do not understand
	\item[DRY]
	    Don't Repeat Yourself
    \end{description}


    % ---------------------------------------------------------------------- %
    % This is where the body of the report begins; usually with an Introduction
    %
    \SANDmain		% Start the main part of the report

    \chapter{Introduction}
	\label{Intro}
	\input{CommonIntro}


    \chapter{A Chapter With Subfigures}
	\input{CommonSubFig}


    \chapter{A Long Chapter}\label{sec:long}
	\input{CommonLong}


    \chapter{Conclusion}
	\input{CommonConclusion}

    \nocite{*}


    % ---------------------------------------------------------------------- %
    % References
    %
    \clearpage
    % If hyperref is included, then \phantomsection is already defined.
    % If not, we need to define it.
    \providecommand*{\phantomsection}{}
    \phantomsection
    \addcontentsline{toc}{chapter}{References}
    \bibliographystyle{plain}
    \bibliography{SANDExample}


    % ---------------------------------------------------------------------- %
    %
    \appendix
    \chapter{Historical Perspective}
	\input{CommonHistory}


    \chapter{Some Other Appendix}
	\input{CommonAppendix}

    % \printindex

    \include{SANDdistribution}

\end{document}
